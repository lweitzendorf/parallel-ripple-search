Pathfinding is a very old problem and many good serial implementations exist.
Important serial algorithms, such as Dijkstra and A*, don't always translate well to a multi-core setting. 
Algorithms that sacrifice path length for performance gains such as fringe search \cite{Fringe} show more promise in this regard.

\mypar{A*} 
The natural serial algorithm to pick for pathfinding is A* \cite{A*}, its use of a heuristic cost function $h(n)$ that estimates the cost from vertex $n$ to the target, makes extending the explored space simple and efficient. 
Common implementations keep a \textit{min-heap} of open vertices, choosing a path that optimizes the cost function $f(n) = g(n) + h(n)$, where $g(n)$ is the actual cost from source to vertex $n$.
This allows A* to explore the unvisited neighboring vertices closest to the target in each iteration, finding the best path in an efficient manner.

\mypar{Parallelized A*} 
Many attempts have been made to parallelize A*. However, due to its reliance on a sorted data structure, there is no efficient way to distribute work among cores. In fact, handling concurrent updates to this structure becomes such a bottleneck that a serial version of A* often outperforms its various parallelizations \cite{parallel-astar}. 

\mypar{Fringe} 
At its core, fringe search is very similar to A*; it merely sacrifices optimality for runtime \cite{Fringe}.
Instead of keeping a sorted data structure that needs to have invariants re-established when the mininum element is removed, fringe simply keeps a list of all vertices in the search boundary. 
The algorithm then explores those that have better heuristics than a certain threshold value.
This saves computational time for exploration by removing the necessity to keep a data structure sorted but still expands the search space in an informed manner.

\mypar{Distributed Fringe} 
A parallel version of fringe search was proposed by S. Brand \cite{distributed-fringe}. As the vertices to be explored are unsorted, work can be distributed over multiple cores. 
However, the author showed many shortcomings of this approach. In particular, the fact that work is typically not distributed evenly across cores. 
In fact, the heuristic restricts the search boundary only in the most promising direction, which can only be followed by a single core at a time.

\mypar{Bidirectional} 
To make use of at least \textit{two} CPU cores when generating a path, one can start any serial algorithm twice: one search from the source towards the target, the other in reverse.
Typically, A* is used for this and the path is recreated from wherever the two searches meet. This is trivially parallelizable and referred to as parallel bidirectional search \cite{PRS}. The obvious shortcoming of this algorithm is that modern architectures have far more than two available cores and the scalability requirement is not met.